\section{Introduction}
\label{intro}
In order to extract information about the 2D structure and composition of the crust from seismic observations, it is necessary to be able to predict how seismic wavefields are affected by complex structures. Since exact analytical solutions to the wave equations do not exist for most subsurface configurations, the solutions can be obtained only by numerical methods. Various techniques for seismic wave modeling in realistic (complex) media have been developed. Explicit finite-difference methods have been widely used to model seismic wave propagation in 2D elastic media, because of their ability to accurately model seismic waves in arbitrary heterogeneous media. The FD software described here is based on the original work of \citet{virieux:86} and \citet{levander:88} who both formulated staggered-grid, finite-difference scheme based on a system of first-order coupled elastic equations where the variables are stresses and velocities. This distribution of wavefield and material parameters is referred to as the standard staggered grid (SSG). \citet{robertsson:94} extended the elastic SSG algorithms of Virieux and Levander to the viscoelastic case. To incorporate absorption he applied the \enquote{generalized standard linear solid} rheological model (GSLS) which was first proposed by \citet{emmerich:87}.

Not only accurate seismic modeling of attenuation is essential for the understanding of wave propagation in the subsurface, but also the modeling of seismic anisotropy. Most real world rocks exhibit a directional dependence of both wave velocities and attenuation. Therefore, in the past decades, anisotropy of seismic velocities and attenuation have been incorporated into seismic modeling and imaging practice (e.g., \cite{Carcione:88}; \cite{Komatitisch:99}; \cite{Thomsen:86}; \cite{Tsvankin:12}; \cite{Bai:16}).

The main drawback of the FD method is that modeling of realistic models consumes vast quantities of computational resources. Such computational requirements are generally beyond the resources for sequential platforms (single PC or workstation). In recent years it has became feasible to use clusters of workstations or PCs for scientific computing. FD modeling can benefit from this technique by using the Message Passing Interface (MPI) (\cite{bohlen:02}). 

Using the free and portable MPI, the calculations are distributed on PCs or workstations which are connected by an in-house network. By clustering a set of processors, for example PCs running Linux, wall-clock times can be decreased and possible grid sizes  can be increased significantly. 