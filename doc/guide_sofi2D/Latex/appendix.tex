\section{Appendix: Parameters used in the code}
\label{parameters}

\begin{verbatim}
MODEL PARAMETERS
    
    general parameters
    
    READMOD=0 ........................ switch to read model parameters from
                                       MFILE
    MFILE[STRING_SIZE]="" ............ model file name
    NX=0 ............................. number of grid points in x-direction
    NY=0 ............................. number of grid points in y-direction
                                       (depth)
    
    source parameters
    
    SOURCE_TYPE=0 .................... type of source
    SOURCE_SHAPE=0 ................... shape of source-signal
    SIGNAL_FILE[STRING_SIZE]="" ...... name of external signal file
    SRCREC=0 ......................... read source parameters from external
                                       source file
    SOURCE_FILE[STRING_SIZE]="" ...... name of source parameter file
    RUN_MULTIPLE_SHOTS=0 ............. multiple shots modeled simultaneously
                                       or individually
    PLANE_WAVE_DEPTH=0.0 ............. depth of plane wave excitation [m]
    PLANE_WAVE_ANGLE=0.0 ............. dip of plane wave from vertical [deg]
    TS=0.0 ........................... duration of source signal [s]
    
    receiver parameters
    
    READREC=0 ........................ switch to read receiver positions from
                                       file
    REC_FILE[STRING_SIZE]="" ......... name of external receiver file
    REFREC[4]={0.0, 0.0, 0.0, 0.0} ... reference point for receiver coordinate 
                                       system
    XREC1=0.0 ........................ x-position of first receiver [m]
    XREC2=0.0 ........................ x-position of last receiver [m]
    YREC1=0.0 ........................ y-position of first receiver [m]
    YREC2=0.0 ........................ y-position of last receiver [m]
    NGEOPH ........................... distance between two adjacent receivers 
                                       [gridpoints]
    REC_ARRAY=0 ...................... number of receivers in 1D receiver array
    REC_ARRAY_DEPTH=0.0 .............. depth of first plane [m] 
    REC_ARRAY_DIST=0.0 ............... increment between receiver planes [m]
    DRX=0............................. increment between receivers in each plane 
                                       [gridpoints]
    attenuation parameters
    
    L=0 .............................. number of relaxation parameters
    FL=0.0 ........................... frequency of each relaxation parameters 
                                       [Hz]
    TAU=0.0 .......................... ratio of retardation and relaxation time
    

IO PARAMETERS
    
    seismogram parameters
    
    SEISMO=0 ......................... switch to output components of seismograms
    NDT=1 ............................ sampling rate of seismograms [timesteps]
    SEIS_FORMAT=0 .................... data output format for seismograms
    SEIS_FILE[STRING_SIZE]="" ........ name of output file of seismograms
    
    snapshot parameters
    
    SNAP=0 ........................... switch to output of snapshots
    SNAP_FORMAT=0 .................... data output format for snapshots
    SNAP_FILE[STRING_SIZE]="" ........ name of output file of snapshots
    TSNAP1=0.0 ....................... first snapshot [s]
    TSNAP2=0.0 ....................... last snapshot [s]
    TSNAPINC=0.0 ..................... increment between snapshots [s]
    IDX=1 ............................ increment in x-direction [gridpoints]
    IDY=1 ............................ increment in y-direction [gridpoints]
    
    other parameters
    
    WRITE_MODELFILES=0 ............... switch to output model files
    SIGOUT=0 ......................... switch to output source wavelet
    SIGOUT_FORMAT=0 .................. data output format for source wavelet
    SIGOUT_FILE[STRING_SIZE]="" ...... name of output file of source wavelet
    LOG=0 ............................ switch to output logging information
    LOG_FILE[STRING_SIZE]="" ......... name of output file of logging information
    LOG_VERBOSITY .................... set how much information is output
    OUTNTIMESTEPINFO=1 ............... information on the time step given to
                                       screen/file
    CHECKPTREAD=0 .................... switch to read wavefield from checkpoint 
                                       file
    CHECKPTWRITE=0 ................... switch to save wavefield to checkpoint
                                       file
    CHECKPTFILE[STRING_SIZE]="" ...... name of checkpoint file

FD PARAMETERS
    
    general parameters
    
    WEQ=0 ............................ wave equation
    DH=0.0 ........................... spacial increment [m]
    FDORDER=0 ........................ spatial FD order 
    TIME=0.0 ......................... modeling time [s]
    DT=0.0 ........................... modeling time increment [s]
    FDORDER_TIME=0 ................... temporal FD order
    
    MPI parameters
    
    MAXRELERROR=0..................... maximum relative group velocity error
    NPROCX=1 ......................... number of processors in x-direction
    NPROCY=1 ......................... number of processors in y-direction
    
    boundary parameters
    
    FREE_SURF=0 ...................... switch to apply free surface at the top 
                                       of the model
    BOUNDARY=0........................ switch to apply periodic boundary condi- 
                                       tion at edges
    ABS_TYPE ......................... type of the absorbing boundary
    FW=0 ............................. width of absorbing frame [gridpoints]
    DAMPING=0.0 ...................... attenuation at the edges of the grid
                                       [%]
    
    PML parameters
    
    NPOWER .......................... exponent for calculation of damping profile
    K_MAX_CPML ...................... 
    FPML ............................ dominant signal frequency (usually FC) [Hz]
    VPPML............................ attenuation velocity within the PML bounda- 
                                      ry [m/s]
\end{verbatim}